% Parameters for ctf estimate
\renewcommand{\arraystretch}{1.2}
\begin{longtable}[l]{| l || p{110mm} |}
\captionsetup{labelfont=bf}
\caption[\code{ctf estimate} parameters]{\code{ctf estimate} parameters. Your parameter file should have the following parameters.\\ \textcolor{myred}{\textbf{*}} indicates the required parameters, \textcolor{blue}{\textbf{*}} indicates expert parameters. Expert parameters should not be changed except if you know what you are doing. The other parameters are optional.}\\

\hline
\multicolumn{2}{|c|}{\textbf{Microscope settings}}\\
\hline

-- \code{VOLTAGE}\textcolor{myred}{\textbf{*}} & Accelerating voltage of the microscope in Volts (e.g. 300e3).\\
-- \code{Cs}\textcolor{myred}{\textbf{*}} & Spherical aberration of the microscope in meters (e.g. 2.7e-6).\\
-- \code{AMPCONT}\textcolor{myred}{\textbf{*}} & Percent amplitude contrast ratio (e.g. 0.09).\\
-- \code{PIXEL\_SIZE}\textcolor{myred}{\textbf{*}} & Pixel size in meters per pixel (e.g. 1.8e-10). Must match the header of the stacks in \code{fixedStacks/*.fixed}.\\
-- \code{SuperResolution}\textcolor{myred}{\textbf{*}} & Whether the stacks are super-sampled. If \code{1}, {\emClarity} will Fourier crop by a factor of 2 and set the actual pixel size to \code{2 * PIXEL\_SIZE}. Note that this is not tested anymore, so it is preferable to Fourier crop the stacks beforehand and set it to 0.\\


% Fiducials
\hline
\multicolumn{2}{|c|}{\textbf{Fiducials}}\\
\hline

-- \code{beadDiameter} & Diameter of the beads to erase, in meters (e.g. 10e-9). This parameter is used if fiducial beads need to be erased, thus only for stacks with a \code{fixedStacks/*.erase} file.\\

-- \code{erase\_beads\_after\_ctf} & Whether or not the fiducial beads should be removed on the raw tilt-series (now) or on the CTF multiplied tilt-series computed during the tomogram reconstruction (section \ref{sec:ctf_3d}). Do not change this option between \code{ctf estimate} and \code{ctf 3d}. Default=\code{false}.\\

% Tilt-scheme
\hline
\multicolumn{2}{|c|}{\textbf{Tilt-scheme}}\\
\hline

-- \code{CUM\_e\_DOSE}\textcolor{myred}{\textbf{*}} & Total exposure in e/\r{A}$^2$.\\
-- \code{doseAtMinTilt}\textcolor{myred}{\textbf{*}} & The exposure each view receive (should be about \code{CUM\_e\_DOSE} / nb of views), in e/\r{A}$^2$.\\
-- \code{oneOverCosineDose}\textcolor{myred}{\textbf{*}} & Whether or not it is a Saxton scheme (dose increase as 1/cos($\alpha$), $\alpha$ being the tilt angle); this will scale \code{doseAtMinTilt} according to the tilt angle (e.g. 0).\\
-- \code{startingAngle}\textcolor{myred}{\textbf{*}} & Starting angle, in degrees (e.g. 0).\\
-- \code{startingDirection}\textcolor{myred}{\textbf{*}} & Starting direction; should the angles decrease or increase (neg or pos).\\
-- \code{doseSymmetricIncrement}\textcolor{myred}{\textbf{*}} & The number of tilts  before each switch in direction. 0=false, 2="normal" dose symmetric. The original dose symmetric scheme included 0 in the first group. For this, specify the number as a negative number.\\

% Defocus estimate
\hline
\multicolumn{2}{|c|}{\textbf{Defocus estimate}}\\
\hline

-- \code{defCutOff}\textcolor{myred}{\textbf{*}} & The power spectrum used by \code{ctf estimate} is considered from slightly before the first zero past the first zero to this cutoff, in meter (e.g. 7e-10).\\
-- \code{defEstimate}\textcolor{myred}{\textbf{*}} & Initial rough estimate of the defocus, in meter. With \code{defWindow}, it defines the search window of defoci.\\
-- \code{defWindow}\textcolor{myred}{\textbf{*}} & Defocus window around \code{defEstimate}, in meter; e.g. if \code{defEstimate} = 2.5e-6 and \code{defWindow} = 1.5e-6, try a range of defocus between 1e-6 to 4e-6.\\
-- \code{deltaZtolerance}\textcolor{blue}{\textbf{*}} & Includes the tiles with defocus equal to that at the tilt-axis $\pm\Delta{Z}$, in meters. See section \ref{sec:algo:defocus_estimate} for more details. Default=50e-9.\\
-- \code{zShift}\textcolor{blue}{\textbf{*}} & Used for the handedness check. Shift the evaluation region ($Z_{tilt-axis}\ \pm\code{deltaZtolerance}$) by this amount. See section \ref{sec:algo:defocus_estimate} for more details. Default=150e-9.\\
-- \code{ctfMaxNumberOfTiles}\textcolor{blue}{\textbf{*}} & Limits the number of tiles to include in the power spectrum. The more tiles, the stronger the signal but the longer it takes to compute the power spectrum. Default=4000.\\
-- \code{ctfTileSize}\textcolor{blue}{\textbf{*}} & Size of the (square) tiles, in meters. Default=680e-10.\\
-- \code{paddedSize}\textcolor{blue}{\textbf{*}} & The tiles are padded to this size, in pixel, in real space before computing the Fourier transform. Should be even, large (compared to the tiles), and preferably a power of 2. Default=768.\\

\hline
\end{longtable}
