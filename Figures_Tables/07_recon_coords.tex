\renewcommand{\arraystretch}{1.2}
\begin{longtable}[r]{| c | l || p{110mm} |}
\captionsetup{labelfont=bf}
\caption{\code{recon/<prefix>\_recon.coords}} \label{tab:recon_coords}\\

\hline
\textbf{Line} & {\tilt} \textbf{parameter} & \textbf{Description}\\
\hline
1 & \cellcolor{lightgray} & \code{<prefix>}; stack prefix.\\
\hline
2 & \cellcolor{lightgray} & Number of sub-regions within this stack.\\
\hline
3 & \code{WIDTH} & Width in X of the first sub-region, in pixel.\\
4 & \code{SLICE} 1 & Starting Y coordinate of the first sub-region. Starts from 0.\\
5 & \code{SLICE} 2 & Ending Y coordinate of the first sub-region. Starts from 0.\\
6 & \code{THICKNESS} & Thickness in Z of the first sub-region, in pixels.\\
7 & \code{SHIFT} 1 & Shift in X the reconstructed slice, in pixel. If it is positive, the slice will be shifted to the right and the output, the first sub-region, will contain the left part of the whole potentially reconstructable area.\\
8 & \code{SHIFT} 2 & Shift in Z the reconstructed slice, in pixel. If it is positive, the slice is shifted upward.\\
\hline
... & ... & Same as line 3 to 8, but for the next sub-regions, if any.\\
\hline
\end{longtable}