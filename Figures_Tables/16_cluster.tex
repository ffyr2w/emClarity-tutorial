\begin{figure}[!htb]  % Stay within section
\captionsetup{labelfont=bf}
\centering

\begin{tikzpicture}[every node/.style={minimum width=1.4cm,minimum height=7mm}]

\node (pcr) {${[\bm{S}_r \bm{V}^{T}_{r}]}_{best}$};
\draw[dashed](pcr.west)--($(pcr.west)+(-3.1,0)$);

\node (pc2) at ($(pcr)+(-1.3,-1.94)$) {${[\bm{S}_2 \bm{V}^{T}_{2}]}_{best}$};
\draw[dashed](pc2.west)--($(pc2.west)+(-1.8,0)$);

\node (pc1) at ($(pc2)+(-1.3,-1.94)$) {${[\bm{S}_1 \bm{V}^{T}_{1}]}_{best}$};
\draw[dashed](pc1.west)--($(pc1.west)+(-0.5,0)$);

% pc all
\matrix (pcall) [draw,matrix of math nodes,fill=white] at ($(pc1)+(+4,-4)$)
{
c_{1,1} & \cdots & c_{1,a+1} & \cdots & c_{1,b+1} & \cdots\\
c_{2,1} & \cdots & c_{2,a+1} & \cdots & c_{2,b+1} & \cdots\\
\cdots & \ddots & \cdots & \ddots & \cdots & \ddots\\
c_{p,1} & \cdots & c_{p,a+1} & \cdots & c_{p,b+1} & \cdots\\
};

\draw[arrow](pcall.south west)--(pcall.south east) node[midway,sloped,below] {$a+b+c$ axes};
\draw[arrow](pcall.north west)--(pcall.south west) node[midway,sloped,below] {$p$ particles};

\node (dima) at ($(pcall)+(-2.75,0)$) [draw,minimum width=2.5cm,minimum height=3.1cm] {};
\draw[arrow](pc1.east)-|($(dima.north)+(0,0)$) node[near end,sloped,above] {$a$ axes};

\node (dimb) at ($(pcall)+(0,0)$) [draw,minimum width=2.5cm,minimum height=3.1cm] {};
\draw[arrow](pc2.east)-|($(dimb.north)+(0,0)$) node[near end,sloped,above] {$b$ axes};

\node (dimc) at ($(pcall)+(2.75,0)$) [draw,minimum width=2.5cm,minimum height=3.1cm] {};
\draw[arrow](pcr.east)-|($(dimc.north)+(0,0)$) node[near end,sloped,above] {$c$ axes};


% kmeans
\matrix (kmeans) [draw,matrix of math nodes,fill=white] at ($(pcall)+(+8,0)$)
{
1 & c(1)\\
2 & c(2)\\
\vdots & \vdots\\
p & c(p)\\
};

\draw[arrow](pcall.east)--(kmeans.west) node[midway,sloped,above] {$k$-means};

\node (tittle_particle) at ($(kmeans)+(-0.75,2.5)$) [rotate=90] {particles};
\node (tittle_class) at ($(kmeans)+(0.64,2.19)$) [rotate=90] {class};


\end{tikzpicture}

\caption[Clustering]{The data is projected onto the principal axes, for each $r$ length scale and stacked into one single $\bm{SV}^T$ matrix of principal components. For visualization, the principal components are transposed to have the axes as columns and the particles as rows. The projected data is then clustered, usually with $k$-means. As a result, each particle is assigned to a class.}
\label{fig:cluster}
\end{figure}
