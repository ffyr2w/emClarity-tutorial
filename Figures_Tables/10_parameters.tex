% Parameters for ctf 3d
\renewcommand{\arraystretch}{1.2}
\begin{longtable}[l]{| l || p{115.5mm} |}
\captionsetup{labelfont=bf}
\caption[\code{ctf 3d} parameters]{\code{ctf 3d} parameters. Your parameter file should have the following parameters.\\ \textcolor{myred}{\textbf{*}} indicates the required parameters, \textcolor{blue}{\textbf{*}} indicates expert parameters. Expert parameters should not be changed except if you know what you are doing. The other parameters are optional.}\\


%% Parameters:
% flgDampenAliasedFrequencies (default=0)

\hline
\multicolumn{2}{|c|}{\textbf{Microscope settings}}\\
\hline

-- \code{PIXEL\_SIZE}\textcolor{myred}{\textbf{*}} & Pixel size in meter per pixel (e.g. 1.8e-10). Must match the header of the stacks in \code{fixedStacks/*.fixed}.\\
-- \code{SuperResolution}\textcolor{myred}{\textbf{*}} & Whether or not the \code{fixedStacks/*.fixed} are super-sampled. Not that this should be the same value you used for \code{ctf estimate} in section \ref{sec:defocus_estimate}.\\

-- \code{Ali\_samplingRate}\textcolor{myred}{\textbf{*}} & Binning factor (1 means no binning) of the output reconstruction.\\

% CTF correction
\hline
\multicolumn{2}{|c|}{\textbf{CTF correction}}\\
\hline

-- \code{useSurfaceFit}\textcolor{blue}{\textbf{*}} & Whether or not the spatial model should be calculated as a function of $x,\ y$ coordinates. If 0, the spatial model is a plane (constant center-of-mass). See section \ref{sec:algo:ctf_3d:spatial_model} for more details.\\

-- \code{flg2dCTF}\textcolor{blue}{\textbf{*}} & Whether or not the CTF correction should correct for the defocus gradients along the electron beam (thickness of the specimen). If 1, only one $z$ section is used. See section \ref{sec:algo:ctf_3d:defocus_step} for more details.\\

\hline
\multicolumn{2}{|c|}{\textbf{Others}}\\
\hline

-- \code{erase\_beads\_after\_ctf} & Whether or not the fiducial beads should be removed before or after CTF multiplication. Default=\code{false}.\\

-- \code{applyExposureFilter} & Whether or not the exposure filter should be applied. If you turn it off, make sure it is turned-off during subtomogram averaging as well.\\

-- \code{super\_sample}\textcolor{blue}{\textbf{*}} & Compute the back projection in a slice larger by the given integer factor (max=8) in each dimension, by interpolating the projection data at smaller intervals ("super-sampling"). This corresponds to the \code{SuperSampleFactor} entry from {\tilt}. Default=\code{0}.\\

-- \code{expand\_lines}\textcolor{blue}{\textbf{*}} & If \code{super\_sample} is greater than 0, expand projection lines by Fourier padding (sync interpolation) when super-sampling, which will preserve higher frequencies better but increase memory needed. This corresponds to the \code{ExpandInputLines} entry from {\tilt}.Default=\code{0}\\

\hline
\end{longtable}

