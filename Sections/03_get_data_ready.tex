\section{Get your data ready}  \label{sec:get_data_ready}

In this tutorial, we will use the 12 tilt-series of the fast-incremental single-exposure (FISE) data deposited on \href{https://www.ebi.ac.uk/pdbe/emdb/empiar/entry/10304/}{EMPIAR-10304} from \cite{eisenstein_2019}. You should be able to get a $\sim$5.9\si{\angstrom} map from this tutorial. To download the motion-corrected tilt-series, it is advisable to use the "Aspera Connect" option or alternatively you can pull directly from the ftp server with the following command:
\begin{lstlisting}
wget -b \
ftp://ftp.ebi.ac.uk/pub/databases/empiar/archive/10304/data/
\end{lstlisting}

\renewcommand{\arraystretch}{1.2}
\begin{longtable}[c]{| l || p{130mm} |}
\hline
-- \textbf{Sample}: & 70S ribosomes, 10nm colloidal gold fiducials.\\
\hline
-- \textbf{Tilt-scheme}: & Dose-symmetric starting from 0\textdegree, 3\textdegree  increment, $\pm$60\textdegree, 120e/\si{\angstrom}$^2$ total exposure.\\
\hline
-- \textbf{Instruments}: & Krios with single-tilt axis holder equipped with a Gatan K3 direct electron detector. 2.1\si{\angstrom}/pix. $\sim$176\textdegree \ image rotation.\\
\hline
\captionsetup{labelfont=bf}
\caption{Tutorial data-set}
\end{longtable}

\begin{note}For this tutorial, we don't necessarily recommend to align the 12 tilt-series manually, as it can be quite redundant, but for beginners, we do recommend to at least try aligning one tilt-series with {\ETomo}. In any case, you should be able to use \code{emClarity autoAlign} for this dataset.
\end{note}
