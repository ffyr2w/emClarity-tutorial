\section{Initialize the project} \label{sec:init}

\subsection{Objectives}
This step is creating the project metadata that will be used throughout the processing. There are three main things that {\emClarity} will do. First, it will grab the sub-region coordinates in \code{/recon/<prefix>.coords}. Second, it will grab the tilt-series CTF estimate stored in \code{fixedStacks/ctf/<prefix>\_ali1\_ctf.tlt}. Lastly, it will grab the particle coordinates from \code{convmap/<prefix>\_<nb>\_<bin>.csv}. As explained in the last section, peaks can be removed from the \code{.csv} files using the corresponding \code{.mod} file.

\begin{note}Once ran, these files are ignored and will not be used again. If one needs to modify some of the above-mentioned information, this step must be re-run for the modification to be effective.
\end{note}


\subsection{Parameters}

% Parameters for init
\renewcommand{\arraystretch}{1.2}
\begin{longtable}[l]{| l || p{114mm} |}
\captionsetup{labelfont=bf}
\caption[\code{init} parameters]{\code{init} parameters. Your parameter file should have the following parameters.\\ \textcolor{myred}{\textbf{*}} indicates the required parameters, the other parameters are optional.}\\

\hline

-- \code{subTomoMeta}\textcolor{myred}{\textbf{*}} & Project name. Most output files will have the project name as prefix and the metadata is saved in a MATLAB file called \code{<subTomoMeta>.mat}.\\ \hline

-- \code{Tmp\_samplingRate}\textcolor{myred}{\textbf{*}} & Sampling (i.e. binning) at which the template matching was ran. It should corresponds to the binning registered in the filename of the .csv and .mod file.\\ \hline

-- \code{fscGoldSplitOnTomos}\textcolor{myred}{\textbf{*}} & Whether or not the particles from the same subregions should be kept in the same half-set or distributed randomly. Two sub-regions are overlapping, you can set this parameter to 1/True to prevent the duplicates to be in the same half-set. However, we do recommend to keep the sub-regions from overlapping and always keep this parameter to 0/False.\\ \hline

-- \code{lowResCut} & This correspond to a rough estimate of the initial resolution of the data-set, directly coming out from the picking. The default value is set to 40\si{\angstrom} and it can be lowered for data-sets collected at low defocus. In most situation, the default value is fine to start with. As the processing goes, {\emClarity} will progressively lower this resolution estimate using the FSC.\\ \hline

\end{longtable}



\subsection{Run}

The \code{init} routine has the following signature:
\begin{lstlisting}
>> emClarity init <param>
\end{lstlisting}
where \code{<param>} is the parameter file name.

\subsection{Outputs}

The main output is of course the output file \code{<subTomoMeta>.mat}. This step should only take a few seconds to run and it will output to the terminal, the total number of particles and the number of particles before and after cleaning, for each sub-region.