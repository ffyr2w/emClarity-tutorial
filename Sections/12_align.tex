\section{Subtomogram alignment} \label{sec:align}

\subsection{Objectives}
Once the reconstructions are generated, we can estimate the $\phi,\ \theta,\ \psi$ rotations and $x,\ y,\ z$ translations between the references and each particle, or in other words, we can maximise the constrained cross-correlation (CCC) between the references and each particle (more details in section \ref{sec:algo:align}). As this procedure directly compares the references with the particles and because the references result from the average of the transformed particles, the subtomogram averaging and alignment steps should be run until convergence of the references.

\subsection{Parameters}
% Parameters for alignRaw
\renewcommand{\arraystretch}{1.2}
\begin{longtable}[l]{| l || p{110.5mm} |}
\captionsetup{labelfont=bf}
\caption[\code{alignRaw} parameters]{\code{alignRaw} parameters. Your parameter file should have the following parameters.\\
\textcolor{myred}{\textbf{*}} indicates the required parameters, \textcolor{blue}{\textbf{*}} indicates expert parameters. Expert parameters should not be changed except if you know what you are doing. The other parameters are optional.}\\

\hline
\multicolumn{2}{|c|}{\textbf{Sampling}}\\
\hline

-- \code{PIXEL\_SIZE}\textcolor{myred}{\textbf{*}} & Pixel size in meter per pixel (e.g. 1.8e-10). Must match the header of the stacks in \code{fixedStacks/*.fixed}.\\
-- \code{SuperResolution}\textcolor{myred}{\textbf{*}} & Whether or not the \code{fixedStacks/*.fixed} are super-sampled. Not that this should be the same value you used for \code{ctf estimate} in section \ref{sec:defocus_estimate}.\\

-- \code{Ali\_samplingRate}\textcolor{myred}{\textbf{*}} & Current bin factor (1 means no binning). The sub-region tomograms at this given binning must be already reconstructed in the \code{cache} directory. You must use the same sampling than during \code{avg}.\\

\hline
\multicolumn{2}{|c|}{\textbf{Mask}}\\
\hline
-- \code{Ali\_mType}\textcolor{myred}{\textbf{*}} & Type of mask; ``cylinder'', ``sphere'', ``rectangle''.\\
-- \code{particleRadius}\textcolor{myred}{\textbf{*}} & [$x,\ y,\ z$] particle radius, in \si{\angstrom}. It should be the smallest values to contain particle. For particles in a lattice, neighboring particles can be used in alignment by specifying a larger mask size, with \code{Ali\_Radius}, but this parameter must correspond to the central unit.\\
-- \code{Ali\_mRadius}\textcolor{myred}{\textbf{*}} & [$x,\ y,\ z$] mask radius, in \si{\angstrom}. The mask size must be large enough to contain the entire particle (i.e. larger than \code{particleRadius}), the delocalized signal, proper apodization, and to avoid wraparound error in cross-correlation.\\
-- \code{Ali\_mCenter}\textcolor{myred}{\textbf{*}} & [$x,\ y,\ z$] shifts, in \si{\angstrom}, relative to the center of the reconstruction. Positive shifts translate the \code{Ali\_mType} mask to the right of the axis.\\

-- \code{Peak\_mRadius} & Further restrict the translations to this radius. By default (0), the translations are limited to the \code{particleRadius}.\\

-- \code{flgCenterRefCOM} & Whether or not the references should be shifted to their center-of-mass before starting the alignment\footnote{For membrane proteins, or phage particles for example, this parameter should be shut off either from the beginning, or after a cycle or two.}. Default=1.\\


-- \code{scaleCalcSize}\textcolor{blue}{\textbf{*}} & Scale the box size (section \ref{sec:algo:avg:box}) by this number. Default=1.5.\\

\hline
\multicolumn{2}{|c|}{\textbf{Symmetry}}\\
\hline

-- \code{Raw\_classes\_odd}\textcolor{myred}{\textbf{*}} & Parameter used to control the C symmetry of the first half set. It should be ``\code{[0; <C>.*ones(2,1)]}'', where \code{<C>} is the central symmetry. This is equivalent to \code{[0; <C>; <C>]}.\\
-- \code{Raw\_classes\_eve}\textcolor{myred}{\textbf{*}} & Parameter used to control the C symmetry of the second half set. It should be identical to \code{Raw\_classes\_odd}.\\
%-- \code{force_no_symmetry} & Turn off the symmetry of the particle. Default=0.\\
-- \code{symmetry} & New parameter used to control the symmetry. CX, I/I2, O, and DX are supported. The old default will be used if this parameter is not specified.\\

\hline
\multicolumn{2}{|c|}{\textbf{Angular search}}\\
\hline

-- \code{Raw\_angleSearch}\textcolor{myred}{\textbf{*}} & Angular search, in degrees. Format is [$\Theta_{out},\ \Delta_{out},\ \Theta_{in},\ \Delta_{out}$]. For example, [$180,\ 15,\ 180,\ 12$], specifies a $\pm$180\textdegree\ out of plane search (polar and azimuth angles) with 15\textdegree\ steps and $\pm$180\textdegree\ in plane search (planar angles) with 12\textdegree\ steps.\\

\hline
\multicolumn{2}{|c|}{\textbf{Others}}\\
\hline

-- \code{subTomoMeta}\textcolor{myred}{\textbf{*}} & Project name. At this step, {\emClarity} excepts to find the metadata \code{subTomoMeta}.mat in the project directory. Most output files will have the project name as prefix.\\

-- \code{Raw\_className}\textcolor{myred}{\textbf{*}} & Class ID for subtomogram averaging and alignment. You should leave it set to zero.\\

-- \code{Cls\_className}\textcolor{myred}{\textbf{*}} & Class ID for classification. You should leave it set to zero for now. For more information, see section \ref{sec:classification}.\\

-- \code{Fsc\_bfactor}\textcolor{myred}{\textbf{*}} & Global B-factor applied to both references. Although it can be a vector, at this stage {\emClarity} will only use the first number as B-factor. See \ref{sec:algo:avg:wiener} for more detail.\\

-- \code{flgClassify}\textcolor{myred}{\textbf{*}} & Whether or not this cycle is a classification cycle. It must be 0 if subtomogram alignment is the next step. More information on this in \ref{sec:classification}.\\

-- \code{use\_v2\_SF3D}\textcolor{blue}{\textbf{*}} & Whether or not the new per-particle sampling function procedure should be used, as opposed to the older ``grouped'' sampling functions. This is the default since {\emClarity} 1.5.1.0. Default=1.\\

% I don't think this one is really supported...?
%-- \code{flgCutOutVolumes}\textcolor{blue}{\textbf{*}} & ... Default=0.\\

\hline
\end{longtable}


\subsection{Run} \label{sec:align:run}

To call the subtomogram alignment procedure:
\begin{lstlisting}
>> emClarity alignRaw <param.m> <cycle_nb>
\end{lstlisting}
where \code{<param.m>} is the parameter file you want to use for this cycle and \code{<cycle\_nb>} is the cycle number. As the alignment needs the reconstructions, one must first run the subtomogram averaging procedure.

The sub-regions are loaded and processed independently from one another during the alignment. Once the particles from a sub-region are aligned, the results from the alignment are saved in the \code{alignResume} directory. {\emClarity} will not re-run the alignment of a particular sub-region if the results for this sub-region are saved in \code{alignResume}. This creates the possibility to run multiple instances of {\emClarity}, each one working on a different set of sub-regions. This is especially useful if you have multiple machines connected to the same storage. {\emClarity} offers two possibilities:
\begin{itemize}
    \item \textbf{Reverse order}: You can run two instances of {\emClarity}, the first one, as usual, will process the sub-regions in a particular order and save the results on a sub-region basis.
\begin{lstlisting}
>> emClarity alignRaw <param.m> <cycle_nb>
\end{lstlisting}
    Then, you can run the same job, but with a negative cycle number:
\begin{lstlisting}
>> emClarity alignRaw <param.m> -<cycle_nb>
\end{lstlisting}
    This second run will not update the metadata and will process the data in the reverse order. The first instance of {\emClarity} will use the results from this run to update the metadata.
    
    \item \textbf{Divide the dataset}: If you want to run more than two instances of {\emClarity}, it is also possible using the following signature.
\begin{lstlisting}
>> emClarity alignRaw <param.m> [<cycle_nb>, <idx>, <jobs>]
\end{lstlisting}
    where \code{<cycle\_nb>} is the cycle number (as usual), \code{<jobs>} is the number of total jobs to split into, which should be less than the number of sub-regions, of course. \code{<idx>} is the index of the current job, from 1 to \code{<total>}. For instance, if you want to share the alignment across 4 GPU nodes and your cycle number is 8, you have to run the following commands.
\begin{lstlisting}
>> emClarity alignRaw <param.m> [8, 1, 4]  # @node1
>> emClarity alignRaw <param.m> [8, 2, 4]  # @node2
>> emClarity alignRaw <param.m> [8, 3, 4]  # @node3
>> emClarity alignRaw <param.m> [8, 4, 4]  # @node4
\end{lstlisting}
    Then, run the usual command as shown below. This should only take a few seconds as it only extracts the results saved in \code{alignResume} by the different jobs and update the metadata.
\begin{lstlisting}
>> emClarity alignRaw <param.m> 8
\end{lstlisting}
    
\end{itemize}


\subsection{Outputs}

The results (relative rotation and shifts, plus the new associated CCC score, per particle) from the alignment are saved in \code{alignResume} and the metadata is updated.

Usually at this point of the workflow, you can either start a new cycle and reconstruct the subtomogram average (section \ref{sec:avg}), the final reconstruction (section \ref{sec:final_map}) or run a classification, or you can stay in the same cycle and run a tilt-series refinement (section \ref{sec:tomoCPR}).