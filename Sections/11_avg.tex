\section{Subtomogram averaging} \label{sec:avg}

\subsection{Objectives}

Once the tomograms are generated, {\emClarity} can calculate the subtomogram averages, one for each half-set. A volume-normalized single-particle Wiener (SPW) filter \cite{volume_normalized_SPW} will minimize the reconstruction error of the particle density by applying an ``optimal'' (low-pass) filter and correcting for all of the systematic changes to the signal imposed by the microscope and image processing algorithms on the reconstructions.

\begin{note}This filter is optimum as long as we correctly estimate the Signal-to-Noise Ratio of the two subtomogram averages (one for each half-set). This estimate depends on the conical Fourier shell correlation described in \myref{sec:algo:avg}.
\end{note}

\subsection{Parameters}

% Parameters for avg
\renewcommand{\arraystretch}{1.2}
\begin{longtable}[l]{| l || p{108.5mm} |}
\captionsetup{labelfont=bf}
\caption[\code{avg} parameters]{\code{avg} parameters. Your parameter file should have the following parameters. \textcolor{myred}{\textbf{*}} indicates the required parameters, \textcolor{blue}{\textbf{*}} indicates expert parameters. Expert parameters should not be changed except if you know what you are doing. The other parameters are optional.} \label{param:avg}\\

\hline
\multicolumn{2}{|c|}{\textbf{Sampling}}\\
\hline

-- \code{PIXEL\_SIZE}\textcolor{myred}{\textbf{*}} & Pixel size in meter per pixel (e.g. 1.8e-10). Must match the header of the stacks in \code{fixedStacks/*.fixed}.\\

-- \code{Ali\_samplingRate}\textcolor{myred}{\textbf{*}} & Current bin factor (1 means no binning). The sub-region tomograms at this given binning must be already reconstructed in the \code{cache} directory. If they aren't, you'll need to run \code{ctf 3d} before running this step.\\

\hline
\multicolumn{2}{|c|}{\textbf{Mask}}\\
\hline
-- \code{Ali\_mType}\textcolor{myred}{\textbf{*}} & Type of mask; ``cylinder'', ``sphere'', ``rectangle''.\\
-- \code{particleRadius}\textcolor{myred}{\textbf{*}} & [$x,\ y,\ z$] particle radius, in \si{\angstrom}. It should be the smallest values to contain particle. For particles in a lattice, neighboring particles can be used in alignment by specifying a larger mask size, with \code{Ali\_Radius}, but this parameter must correspond to the central unit.\\
-- \code{Ali\_mRadius}\textcolor{myred}{\textbf{*}} & [$x,\ y,\ z$] mask radius, in \si{\angstrom}. The mask size must be large enough to contain the entire particle (i.e. larger than \code{particleRadius}), the delocalized signal, proper apodization, and to avoid wraparound error in cross-correlation.\\
-- \code{Ali\_mCenter}\textcolor{myred}{\textbf{*}} & [$x,\ y,\ z$] shifts, in \si{\angstrom}, relative to the center of the reconstruction. Positive shifts translate the \code{Ali\_mType} mask to the right of the axis.\\
-- \code{scaleCalcSize}\textcolor{blue}{\textbf{*}} & Scale the box size (section \ref{sec:algo:avg:box}) by this number. Default=1.5.\\

\hline
\newpage

\hline
\multicolumn{2}{|c|}{\textbf{Symmetry}}\\
\hline

-- \code{Raw\_classes\_odd}\textcolor{myred}{\textbf{*}} & Parameter used to control the C symmetry of the first half set. It should be ``\code{[0; <C>.*ones(2,1)]}'', where \code{<C>} is the central symmetry. This is equivalent to \code{[0; <C>; <C>]}.\\
-- \code{Raw\_classes\_eve}\textcolor{myred}{\textbf{*}} & Parameter used to control the C symmetry of the second half set. It should be identical to \code{Raw\_classes\_odd}.\\

-- \code{symmetry} & New parameter used to control the symmetry. CX, I/I2, O, and DX are supported. The old default will be used if this parameter is not specified.\\

\hline
\multicolumn{2}{|c|}{\textbf{Fourier shell correlation}}\\
\hline

-- \code{flgCones}\textcolor{myred}{\textbf{*}} & Whether or not {\emClarity} should calculate the conical FSCs. This will greatly impact the calculated half-maps if you have preferential orientation in the data-set. We recommend to leave this to 1 (true) throughout the workflow.\\

-- \code{minimumParticleVolume}\textcolor{blue}{\textbf{*}} & Defines a minimum value for the $f_{mask}/f_{particle}$ ratio (section \ref{sec:algo:avg:molecular_mask} and \ref{sec:algo:avg:fsc}). Default=$0.1$\\

-- \code{flgFscShapeMask}\textcolor{blue}{\textbf{*}} & Apply a very soft mask based on the particle envelope before calculating the FSC (section \ref{sec:algo:avg:molecular_mask}). We highly recommend to not turn this off. Default=1.\\

-- \code{shape\_mask\_test} & Exit after saving the shape mask in \code{FSC/}. This is useful when testing the following two parameters. Default=0.\\

-- \code{shape\_mask\_lowpass} & Low-pass cutoff, in \si{\angstrom}, to apply to the median filtered raw subtomogram averages. Default=14.\\

-- \code{shape\_mask\_threshold} & Initial threshold used for the dilation. If you have extra "dust" outside your particle, it can be helpful to decrease the resolution used in the inital thresholding, i.e. increase \code{shape\_mask\_lowpass}, or to increase this threshold (or both). Default=2.4.\\

\hline
\multicolumn{2}{|c|}{\textbf{Others}}\\
\hline

-- \code{subTomoMeta}\textcolor{myred}{\textbf{*}} & Project name. At this step, {\emClarity} excepts to find the metadata \code{subTomoMeta}.mat in the project directory. Most output files will have the project name as prefix.\\

-- \code{Raw\_className}\textcolor{myred}{\textbf{*}} & Class ID for subtomogram averaging and alignment. You should leave it set to zero.\\

-- \code{Fsc\_bfactor}\textcolor{myred}{\textbf{*}} & Global B-factor applied to both references. See \ref{sec:algo:avg:wiener} for more detail. It can be a vector. In this case, if it is an intermediate reconstruction (i.e. reconstructing half-maps with \code{RawAlignment}) only the first value is used, if it is a final reconstruction (i.e. \code{FinalAlignment} introduced in section \ref{sec:final_map}), {\emClarity} will calculate one reconstruction per value registered in this vector.\\

-- \code{flgClassify}\textcolor{myred}{\textbf{*}} & Whether or not this cycle is a classification cycle. It must be 0 if subtomogram alignment is the next step. More information on this in \ref{sec:classification}.\\

-- \code{flgCutOutVolumes}\textcolor{blue}{\textbf{*}} & Whether or not each transformed particle (rotated and sifted) should be saved to \code{cache} directory. Note that the subtomogram have an extra padding of 20 pixel. This makes the reconstruction much slower if activated. Default=0.\\

-- \code{flgQualityWeight}\textcolor{blue}{\textbf{*}} & The particles with an alignment score (Constrain Cross-Correlation score, CCC) below the average CCC score of the entire data-set are down-weighted before being added to the reference. This parameter regulates the strength of the weighting.\\

-- \code{flgCCCcutoff}\textcolor{blue}{\textbf{*}} & Particles with an alignment score (Constrain Cross-Correlation score, CCC) below this value are ignored from the reconstruction. Default=0.\\

-- \code{use\_v2\_SF3D} & Whether or not the new per-particle sampling function procedure should be used, as opposed to the older ``grouped'' sampling functions. This is the default since {\emClarity} 1.5.1.0. Default=1.\\

%%%%% Optional
% flgCutOutVolumes
% loadTomo: is it useful? so it load the entire tomo in memory... mouais.

\hline
\end{longtable}


\subsection{Run}

To call the subtomogram averaging procedure:
\begin{lstlisting}
>> emClarity avg <param.m> <cycle_nb> RawAlignment
\end{lstlisting}

where \code{<param.m>} is the parameter file you want to use for this cycle, \code{<cycle\_nb>} is the cycle number, starting from 0. Each cycle starts with this step and is usually followed by the subtomogram alignment procedure. \code{RawAlignment} indicates to {\emClarity} that we want to calculate the subtomogram average of the entire half-set. As we'll see later, it is also possible to compute one average (one cluster) for each class.

If you suspect some densities are being omitted by the molecular mask or if you see extra `dust' outside your particle, you can change with the \code{shape\_mask\_*} parameters. If you want to only calculate the mask, you can run the following, with \code{shape\_mask\_test=1}:
\begin{lstlisting}
>> emClarity fsc <param.m> <cycle_nb> RawAlignment
\end{lstlisting}
The same molecular mask is used for averaging and alignment. The values that you use for the averaging will be used during the alignment.

\newpage
\subsection{Outputs}

The two subtomogram averages (half-maps) are saved into the project directory, as:\\ \code{cycleXXX\_projectName\_class0\_REF\_EVE.mrc}, and\\ \code{cycleXXX\_projectName\_class0\_REF\_ODD.mrc}.

The FSC is saved in \code{FSC/cycleXXX\_projectName\_Raw-1-fsc\_GLD.pdf}. The corresponding \code{.txt} file constains the FSC values. The first column is the resolution, in \r{A}\textsuperscript{-1}. The second column contains the Correlation Coefficients (CCs). The other 36 columns contain the CCs for each FSC cones.

For more details about the outputs, see \ref{sec:algo:avg}.